\documentclass[12 pt, a4paper]{article}
\usepackage[norsk]{babel}  								% For norsk oppsett
\usepackage[utf8]{inputenc}
\usepackage{amsmath}
\usepackage{amssymb}
\usepackage{graphicx}
\usepackage{subcaption}
\usepackage{hyperref}
\usepackage{fancyhdr}
\usepackage{enumerate}
\usepackage{float}
\usepackage{tikz}
\usepackage{circuitikz}
\usepackage{physics}
\usepackage[includeheadfoot, margin =1cm]{geometry}
%\usepackage{python}
\usepackage[version=3]{mhchem}
\usepackage{siunitx}
\usepackage{todonotes}
\usepackage{xcolor}
\usepackage{lastpage}
\usepackage{listings}
\renewcommand{\exp}[1]{\mathrm{e}^{#1}}

\lstset{basicstyle=\ttfamily,
  showstringspaces=false,
  commentstyle=\color{red},
  keywordstyle=\color{blue}
}

\usepackage[bottom]{footmisc}
\renewcommand\footnoterule{\rule{\linewidth}{0.5pt}}

\setlength{\parindent}{0cm}

\author{Erik Skaar\\ erikfsk@uio.no}





\begin{document}
\pagebreak


\pagestyle{fancy}
\fancyhf{}
\rhead{FYSMENA4111}
\lhead{Erik Skaar}
\fancyfoot[CE,LO]{\leftmark}
\fancyfoot[LE,RO]{Page \number\value{page} of \pageref{LastPage}}

\renewcommand{\headrulewidth}{2pt}
\renewcommand{\footrulewidth}{1pt}

\section{}%1 



%\begin{align*}
%&n \qquad &2^n - (-1)^n\\
%&n+1 \qquad &2^{n+1} - (-1)^{n+1} \\
%& &= 2(2^{n}) - (-1)^{n+1}\\
%& &= 2(2^{n} + (-1)^n  + (-1)^{n+1}) - (-1)^{n+1}\\
%& &= 2(2^{n} + (-1)^n  - (-1)^{n}) - (-1)^{n+1}\\
%& &= 2(2^{n}- (-1)^{n}) + 2(-1)^n  + (-1)^{n}\\
%& &= 2(2^{n}- (-1)^{n}) + 3(-1)^n \\
%\end{align*}



% \begin{figure}[H]
%     \centering
%     \begin{subfigure}{0.5\textwidth}
%         \centering
%         \includegraphics[width=\linewidth]{result/bilder/Tc/e-Tc}
%         \caption{}
%     \end{subfigure}%
%     ~ 
%     \begin{subfigure}{0.5\textwidth}
%         \centering
%         \includegraphics[width=\linewidth]{result/bilder/Tc/m-Tc}
%         \caption{}
%     \end{subfigure}
%     \caption{a) Shows how E behaves around $T_C$ b) Shows how |M| develops near $T_C$.}
%     \label{fig:tc-E-M}
% \end{figure}






% \begin{center}
% \label{tab:states-2x2-summary}
% \captionof{table}{The table shows a summary from table \ref{tab:states-2x2}. }
% \begin{tabularx}{\textwidth}{c X c X c X c}
%     \hline 
%     \hline 
%         Number of $\color{red}{\uparrow}$ && Multiplicity && Energy && Magnetic moment \\ 
%     \hline
%         4   &&      1      &&      -8J     &&       4       \\  
%         3   &&      4      &&      0J      &&       2       \\
%         2   &&      2      &&      8J      &&       0       \\
%         2   &&      4      &&      0J      &&       0       \\
%         1   &&      4      &&      0J      &&       -2      \\
%         0   &&      1      &&      -8J     &&       -4      \\
%     \hline
% \end{tabularx}
% \end{center}













%\begin{tabular}{|c|c|c|c|c|c|c|}
%	\hline 
%	n & General & Specific & LU & fastest & slowest & $\frac{slowest}{fastest}$\\ 
%	\hline
%	10 & 6.5e-05 & 5e-06 & 4e-05 & Specific & General & 13.0\\ 
%	\hline 
%	100 & 7.5e-05 & 8e-06 & 0.0023 & Specific & LU & 287.5\\ 
%	\hline 
%	1000 & 0.00014 & 4e-05 & 0.26 & Specific & LU & 6500\\ 
%	\hline
%	10000 & 0.0007 & 0.0005 & 142.5 & Specific & LU & 285000 \\ 
%	\hline
%\end{tabular}







%\begin{figure}[H]
%		\centering
%		\includegraphics[width=0.7\linewidth]{ab.png}
%		\caption{Atomene er gule kuler, de elementære vektorene er blå og a vektorene er grønne.}
%		\label{fig:ab}
%\end{figure}



%\begin{figure}[H]
%		\centering
%		\includegraphics[width=0.7\linewidth]{ab.png}
%		\caption{Atomene er gule kuler, de elementære vektorene er blå og a vektorene er grønne.}
%		\label{fig:ab}
%\end{figure}
%\printbibliography






%\begin{figure}[H]
%    \centering
%    \begin{subfigure}{0.5\textwidth}
%        \centering
%        \includegraphics[width=\linewidth]{maybe}
%        \caption{}
%    \end{subfigure}%
%    ~ 
%    \begin{subfigure}{0.5\textwidth}
%        \centering
%        \includegraphics[width=\linewidth]{maybe2}
%        \caption{}
%    \end{subfigure}%
%    \caption{a) is probably wrong and b) is right? If not i am not sure what to comment. No matter what. I am not sure what to comment. Cause b) should be wrong. The code above produces a). Setting kb to 1 gives b)}
%\end{figure}








%\begin{align*}
%	J (x,y)= 
%	\begin{bmatrix}
%		12x^2 & 1\\ 
%		1 & 2
%	\end{bmatrix}
%\end{align*}



\end{document}
